\chapter{Introduction}
\label{ch:intro}
The Maximum Agreement Subtree problem (MAST) provides mutual information between rooted trees, and is defined as such: Given two rooted trees, T1 and T2, created over the same leaf-set \{1,2,3...,n\}, determine the largest possible subset of leaves inducing an agreeing subtree of T1 and T2. For a set of leaves to induce an agreeing subtree for T1 and T2, the subtrees restricted to the set of leaves must be isomorphic, implying structural equivalency.
\\

Let us start by motivating the interest in MAST by giving an example of its application. Suppose that we are interested in inspecting the relationship between DNA obtained from different animal species. This is typically done by the use of  Hierachical Clustering (REF) or Neighbor Joining (REF) to construct evolutionary trees. However, finding the true evolutionary tree is often an elusive task, and evidence is required to support any suggested tree topology. Finding the Maximum Agreement Subtree will present the information that both trees agree on, which makes the information more reliable, given that multiple sources support it. 

The MAST problem applies to all trees, but we will choose to focus on the rooted, binary trees given that the motivation for the problem is primarily rooted in biology and linguistics, where these trees are most common. Specifically, we will examine, implement and compare three different algorithms for solving the MAST problem for binary trees in order to clarify strengths and weaknesses of each in theory and in practice.     



The MAST problem is one of several ways of defining tree distances. -which one is superior?


\begin{align*}
	\frac{3x + y}{7} &= 9  && \text{given}   \\
	3x + y &= 63           && \text{multiply by 7}   \\
	3x &= 63 - y           && \text{subtract y}   \\
	x &= 21 - \frac{y}{3}  && \text{divide by 3}   \\
\end{align*}




\begin{algorithm}
	\caption{Backtracking}\label{euclid}
	\begin{algorithmic}[1]
		\Procedure{RecBackTrack(Tree Tree1, Tree Tree2)}{}
		
		\State $\textit{stringlen} \gets \text{length of }\textit{string}$
		
		\If {$\textit{Type(Tree1) = Type(Tree2) = InternalNode} $} \Return \textit{Combine(RecBackTrack(Tree1.Child1, Tree2.Child1), RecBackTrack(Tree1.Child1, Tree2.Child1))}
		\EndIf
		\State $j \gets \textit{patlen}$
		
		
		
		
		\State $\textit{stringlen} \gets \text{length of }\textit{string}$
		\State $i \gets \textit{patlen}$
		\BState \emph{top}:
		\If {$i > \textit{stringlen}$} \Return false
		\EndIf
		\State $j \gets \textit{patlen}$
		\BState \emph{loop}:
		\If {$\textit{string}(i) = \textit{path}(j)$}
		\State $j \gets j-1$.
		\State $i \gets i-1$.
		\State \textbf{goto} \emph{loop}.
		\State \textbf{close};
		\EndIf
		\State $i \gets i+\max(\textit{delta}_1(\textit{string}(i)),\textit{delta}_2(j))$.
		\State \textbf{goto} \emph{top}.
		\EndProcedure
	\end{algorithmic}
\end{algorithm}


\begin{equation}
	F(N) \le Max\{A + Blog(j) + F(j-1) + F(N-j); 1 \le j \le (N+1)/2)\}	
\end{equation}
Initially it is not obivous that this equation grows at most linearly with N. As such, we need to prove that it does. For this reason, we first investigate the function F by defining a secondary function G that is defined over the constants of F: A, B and F(0).
	\begin{equation*}
	\begin{aligned}
	G(N)=
	\begin{cases}
	F(0) & \text{if N=0}
	\\
	Max\{A + Blog(j) + F(J-1) + F(N-J; 1 \le j \le (N+1)/2)\} & \text{if N>0}             
	\end{cases}
	\end{aligned}
	\phantom{\hspace{6cm}}
	\end{equation*}
From this definition it follows that $G(1)= A + 2G(0)$, and $G(2)=A + G(0) + G(1)$. In this manner we can investigate the first few terms of G.

\begin{eqnarray*}
	G(0) &=& F(0) \\
	G(1) &=& A + 2F(0) \\
	G(2) &=& 2A + 3F(0) \\
	G(3) &=& max\{3A + 4F(0), 3A + Blog(2) + 4F(0)\} \\
	     &=& 3A + Blog(2) + 4F(0) \\
	G(4) &=& max\{4A + 5F(0), 4A + Blog(2) + 5F(0)\} \\
	     &=& 4A + Blog(2) + 5F(0)
\end{eqnarray*}
From these first few terms it quickly becomes apparent that the coefficient of A is N, and the coefficient of F(0) is N+1. However, the coefficient of B is not clear. To determine the growth of B, quite a few more terms are needed to gain a clear picture. We will not include all of those here, but simply state that some investigative work revealed the coefficient of B to be $Nlog(2)-log(N+1)$. It is not immediately apparent that this coefficient is in fact at most linear in complexity. However, the following limit clear shows that it is indeed the case.
$$\lim_{N\to\infty} \frac{Nlog(2)-log(N+1)}{N} = log(2)$$
This means that if the claims we made about the coefficients of A, B and F(0) are true, then the following inequality will show that F(N) can at most grow linearly as a function of N.
\begin{equation}
	F(N) \le AN + F(0)(N+1) + B(Nlog(2)-log(N+1)) 	
\end{equation}
Let us now prove that this inequality indeed holds.
\subsection{Proof}
We prove inquality (1.2) by the use of strong induction. Strong induction consists of a base case, and an inductive case. The base case proves the inequality for a starting point, and the inductive case proves that whenever the inequality holds for a set of sequential integers starting from the base case, then it will also hold for the next integer. Together the base case and inductive case induce the inquality to hold for all N. 
\subsubsection{Base case}
Our base case is defined for the lowest value, N=0. We simply insert this value into the inequality.
\begin{align*}
	F(0)\ \le&\ \  A*0 + F(0)(0+1) + B(0*log(2)-log(0+1)) & \Rightarrow \\
	F(0)\ \le&\ \  0 + F(0) + B(0-0) & \Rightarrow \\
	F(0)\ \le&\ \  F(0) &
\end{align*}
Clearly, the inequaliity holds in the base case.

\subsubsection{Inductive case}
We are now given an Integer N, and we assume in the spirit of strong induction our Induction Hypothesis to be
\begin{equation}
	\forall x \le N,\ F(x) \le Ax + F(0)(x+1) + B(xlog(2)-log(x+1)) 	
\end{equation}
and use that to prove that 
\begin{equation}
	F(N) \le AN + F(0)(N+1) + B(Nlog(2)-log(N+1)) 	
\end{equation}
Let J be the integer that maximizes $A + Blog(j) + F(j-1) + F(N-j)$ in inequality (1.1). From the Induction Hypothesis (1.3), we now have for $x=J-1$ and $x=N-J$ the following two inequalities.
\begin{equation}
	F(J-1) \le A(J-1) + F(0)((J-1)+1) + B((J-1)log(2)-log((J-1)+1)) 	
\end{equation}
\begin{equation}
	F(N-J) \le A(N-J) + F(0)((N-J)+1) + B((N-J)log(2)-log((N-J)+1)) 	
\end{equation}
We can now insert (1.5) and (1.6) into (1.4), creating a huge inequality to verify.
\begin{align*}
	F(N)\ \le&\ \  A + Blog(J) + F(J-1) + F(N-J) \le   & \\
	         &\ \  AN + F(0)(N+1) + B(Nlog(2)-log(N+1) & 
\end{align*}
We expand $F(J-1)$ and $F(N-J)$, and simplify the inequality.

\begin{equation*}	
	\begin{aligned}
	&A + Blog(J) + A(J-1) + F(0)((J-1)+1) +                  & \\
	&B((J-1)log(2)-log((J-1)+1)) + A(N-J) + F(0)((N-J)+1)    & \\
	& + B((N-J)log(2)-log((N-J)+1))                          &\le  \\
	&AN + F(0)(N+1) + B(Nlog(2)-log(N+1))                    & \\
	& & \\
	&\Rightarrow (\text{Add up coefficients of A,B and F(0)})& \\
	& & \\
	&AN + B((N-1)log(2)-log(N-J+1)) + F(0)(N+1)              &\le \\
	&AN + F(0)(N+1) + B(Nlog(2)-log(N+1))                    &  \\	
	& & \\
	&\Rightarrow (\text{Subtract AN and F(0)(N+1)})          & \\
	& & \\	
	&B((N-1)log(2)-log(N-J+1))                               &\le \\
	&B(Nlog(2)-log(N+1))                                     &  \\	
	& & \\
	&\Rightarrow (\text{Divide with B, and subtract (N-1)log(2)})      & \\
	& & \\
	&-log(N-J+1)                                             &\le \\
	&log(2)-log(N+1)                                         &  \\	
	& & \\
	&\Rightarrow (\text{Add log(N+1) and apply log rule})    & \\
	& & \\			
	&log(\frac{N+1}{N-J+1}) \  \le \  log(2)                 & 	
	\end{aligned}
\end{equation*}
We know from (1.1) that $1 \le J \le \frac{N+1}{2}$, and that $J=\frac{N+1}{2}$ maximizes $log(\frac{N+1}{N-J+1})$. If the inquality holds for the maximum J, then it will hold for the entire interval: 
\begin{equation*}	
	\begin{aligned}	
	&log(\frac{N+1}{N-\frac{N+1}{2}+1}) = log(2) \  \le \  log(2);  & N \ge 0 	
	\end{aligned}
\end{equation*}
As such, the inductive case in (1.4) is proven to hold, which completes the proof. $\qed$


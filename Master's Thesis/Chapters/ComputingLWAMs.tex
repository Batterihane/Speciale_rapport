Computing the LWAM for the graph $G(x)$ is done by processing each node of $L(x)$ in order starting with the bottom node. Recall that we will compute the LWAM containing only edges from $u_i$ and nodes below $u_i$ in $L(x)$, for each $u_i \in L(x)$. This is done by adding that information to $\mathcal{T}$ while processing the nodes of $L(x)$. 

First, we will describe the information that needs to be stored in $\mathcal{T}$. Next, we will explain how nodes are processed and how to extract the LWAMs from $\mathcal{T}$. Finally we explain the total time complexity of this step.

\subsubsection{Auxiliary Information in $\mathcal{T}$}
While processing the edges of $G(x)$ we will maintain information at each node of $\mathcal{T}$ about the LWAM of the currently processed edges. For a node $z \in \mathcal{T}$, we define $anc(z)$ as the set of ancestors of $z$, including $z$ itself in order top to bottom. $lfringe(z)$ is defined as the left children of the vertices in $anc(z)$ which are not in $anc(z)$ themselves. $lfringe$ will only contain nodes with indices less than $index(z)$. $rfringe(z)$ is defined analogously where the nodes all have indices greater or equal to $index(z)$. $anc(z)$ need not be stored in $\mathcal{T}$, but can be created while searching for $z$ in $\mathcal{T}$. Likewise, $lfringe(z)$ and $rfringe(z)$ can be created by iterating through $anc(z)$. After processing an edge $(u,v)$, we say that the edge is in $\mathcal{T}$. For a node $z \in \mathcal{T}$ we also say that the edge $(u,v)$ is in $T(z)$ if $T(z)$ contains the node corresponding to $v$.

The following information is maintained at each node $z \in \mathcal{T}$. This is taken directly from \cite{nlogn}, but slightly elaborated.
\begin{itemize}
	\item $g(z)$: After updating $g(z)$, it will hold the heaviest green edge in $\mathcal{T}$ which	forms a proper crossing with each red edge in $\mathcal{T}(z)$.
	\subitem However, $g(z)$ is not maintained at all times, but the correct edge is always $max_{z' \in anc(z)}g(z')$.
	\subitem If there are no red edges in $\mathcal{T}(z)$ when updating $g(z)$, then $g(z)$ will be the heaviest green edge in $\mathcal{T}(z')$, for all $z' \in rfringe(z)$.
	\item $x(z)$: This is the largest weight proper crossing, that is not a single red edge, among the edges in $\mathcal{T}(z)$.
	\item $m(z)$: This is the largest weight agreement matching in $\mathcal{T}$ containing a white edge such that the topmost white edge is in $\mathcal{T}(z)$.
	\item $y(z)$: This the largest weight proper crossing, which is not a single edge, such that the green edge is in $\mathcal{T}$ but not in $\mathcal{T}(z)$, the red edge is in $\mathcal{T}(z)$, and the green edge does not form a proper crossing with each of the red edges in $\mathcal{T}(z)$.
	\item $r(z)$: This is simply the heaviest red edge in $\mathcal{T}(z)$.
\end{itemize}

\subsubsection{Processing Nodes}
For each node $u_i \in L(x)$, starting from the bottommost, we need to process each edge $e = (u_i, v_j)$ incident on it. This is done by searching for the leaf corresponding to $v_j$ in $\mathcal{T}$ and updating information at nodes in $\mathcal{T}$ so that $e$ is contained in $\mathcal{T}$. Leaves are searched for as described in section \ref{st_searching}, where a standard search is used if $d_x(u_i) = 1$ and a sequential search is used if $d_x(u_i) > 1$.

In the following, we will explain how each colour edge is processed when $d_x(u_i) = 1$ and afterwards when $d_x(u_i) > 1$. For a green edge $g$ and a red edge $r$, $g+r$ will refer to the proper crossing between $g$ and $r$.

Let $e=(u_i,v_j)$ be the edge in $G(x)$ that should be processed and $l$ be the leaf in $\mathcal{T}$ corresponding to $v_j$. We know that all edges incident on nodes below $u_i$ in $L(x)$ have already been processed.

\subsubsection{Processing white singleton edges}
The only values that might change in $\mathcal{T}$ when adding a white edge is $m(z), z \in anc(l)$. First, $anc(l)$ is found while searching for $l$ in $\mathcal{T}$ and $rfringe(l)$ is subsequently created. Now the LWAM with $e$ as topmost edge needs to be determined. This is done by finding the LWAM among the edges in $\mathcal{T}$ containing only edges dominated by $e$ and then adding $e$ as the topmost white edge. Since the nodes of $L(x)$ are processed starting from the bottom and $e$ is a singleton edge, we know that $\mathcal{T}$ does not contain any edges incident on $u_i$ or nodes above $u_i$ in $L(x)$. The subtrees of nodes in $rfringe(l)$ contain exactly the edges in $\mathcal{T}$ only incident on nodes below $v_j$ in $R(x)$, so these are the subtrees that should be considered. %By the order we process the edges, we also make sure that the edges from $u_i$ are processed such that $rfringe(l)$ contains no edge incident on $u_i$.
There are four possible cases for the LWAM:
\begin{itemize}
	\item The LWAM contains one or more white edges.
	\subitem This matching is given by $max_{z \in rfringe(l)} m(z)$
	\item The LWAM is a single green edge or a green-red crossing where both edges are in the subtree $\mathcal{T}(z)$ for some $z \in rfringe(l)$.
	\subitem This matching is given by $max_{z \in rfringe(l)} x(z)$
	\item The LWAM is a single red edge or a green-red crossing where the red edge is in the subtree $\mathcal{T}(z)$ for some $z \in rfringe(l)$. The green edge forms a proper crossing with all red edges in $\mathcal{T}(z)$.
	\subitem This matching is given by
	\subsubitem $max_{z \in rfringe(l)} (max_{z' \in anc(z)} g(z')) + r(z)$, if the edge $r(z)$ exists
	\item The LWAM is a green-red proper crossing where the red edge is in the subtree $\mathcal{T}(z)$ for some $z \in rfringe(l)$. The green edge is not in $\mathcal{T}(z)$ and does not form a proper crossing with all red edges in $\mathcal{T}(z)$.
	This matching is given by $max_{z \in rfringe(l)} y(z)$
\end{itemize}

These LWAMs can be found doing a single iteration through rfringe(l) starting from the top. The four cases cover all possible LWAMs containing only edges dominated by $e$, so adding $e$ as the topmost white edge to the largest of these agreement matchings, gives us the LWAM that should replace $m(z), z \in anc(l)$ if it has larger weight than the agreement matching already stored.

In order to avoid copying the agreement matching just found, which would take linear time w.r.t. the number of edges, the new agreement matching can be stored as the white edge $e$ and a pointer to the other agreement matching. This will also minimize the amount of space needed.

\subsubsection{Processing red singleton edges}
The values that might change in $\mathcal{T}$ when adding a red edge is $y(z), g(z)$ and $r(z), z \in anc(l)$ and $g(z'), z' \in lfringe(l) \cup rfringe(l)$. Since none of the green edges in $\mathcal{T}$ can form a proper crossing with $e$, $y(z)$ might change for $z \in anc(l)$. So $y(z)$ is updated:

\begin{equation*}
\begin{aligned}
y(z)=max
\begin{cases}
y(z)
\\
(max_{z' \in anc(z)} g(z')) + r(z) & \text{if both edges exist}             
\end{cases}
\end{aligned}
\phantom{\hspace{6cm}}
\end{equation*}

For the same reason $g(z)$ should be reset for all $z \in anc(l)$. However, in order to not lose the information at g(z), $g(z')$ needs to be updated to $max_{z'' \in anc(z')} g(z'')$ for each $z' \in lfringe(l) \cup rfringe(l)$ before resetting. Finally, $r(z)$ need to be updated for $z \in anc(l)$ if the weight of $e$ is greater than that the current value of $r(z)$. $r(z)=max\{r(z), e\}$.

Updating all these values only requires a single iteration through $anc(l)$ starting from the root, where both the nodes of $anc(l)$ and children of these nodes are updated.

\subsubsection{Processing green singleton edges}
Adding a green edge to $\mathcal{T}$ might change the values $g(z), z \in lfringe(l)$ and $x(z), z \in anc(l)$. When adding the green edge $e$, it will form proper crossings with all red edges in $\mathcal{T}$ incident on nodes in $R(x)$ above $v_j$. This is true since each red edge in $\mathcal{T}$ is incident on a node in $L(x)$ below $u_i$. Therefore $g(z), z \in lfringe(l)$ is updated to $e$ if the weight of $e$ is greater than that of the currently heaviest green edge forming a proper crossing with all red edges in $\mathcal{T}(z)$. $g(z)=max\{e, max_{z' \in anc(z)} g(z')\}, z \in lfringe(l)$. The value $x(z), z \in anc(l)$ might also need to be updated if $e$ or a proper crossing between $e$ and a red edge in $\mathcal{T}(z)$ is heavier than the current value of $x(z)$. $x(z)=max\{x(z), e + max(r(z'))\}, z \in anc(l)$. Here $max(r(z'))$ maximizes over the vertices $z' \in lfringe(l)$ which are descendants of $z$.

The values $g(z), z \in lfringe(l)$ are updated during a single iteration through $anc(l)$ starting at the root. The values $x(z), z \in anc(l)$ are updated during a single iteration through $anc(l)$ starting at $l$.

\subsubsection{Processing nodes with degree > 1}
Recall that we process the edges of $G(x)$ by iterating through $L(x)$ starting from the bottommost node, and process the edges incident on each of these nodes. For a node $u_i \in L(x)$ we will start by processing the white edges incident on $u_i$ in order, starting from the topmost edge, i.e. the edge incident on the topmost node in $R(x)$. The reason for doing top-down processing is explained later. For these edges, the leaves in $\mathcal{T}$ that needs to be searched for, are found with a sequential search as described in section \ref{st_searching}.

Next, we process the red and green edges incident on $u_i$ in order, starting from the bottommost edge. This ensures that when processing a green edge $e=(u_i,v_j)$, no red edges incident on $u_i$ and on a node in $R(x)$ above $v_j$ has been processed. Searching for leaves in $\mathcal{T}$ is also done with sequential search.

Processing edges from non-singleton nodes using sequential search is important for the runtime of this step. Let ${l_1, l_2, ..., l_k}$ be the leaves in the search tree that are searched for when processing $u_i \in L(x)$. The only nodes in the search tree that need to be updated are nodes from the set ${anc(l_1) \cup anc(l_2) \cup ... \cup anc(l_k)}$, but in order to ensure that only a constant amount of work is performed at each node, we need a slightly different approach than for the singleton edges. This was not covered by Cole et. al. \cite{nlogn}, so we came up with our own approach.

In the following, we will explain how each colour edge is processed when $d_x(u_i) > 1$. Let $e_1, e_2, ...$ be the edges in $G(x)$ incident on $u_i$ and $l_1, l_2, ...$ be the corresponding leaves in $\mathcal{T}$ that should be searched for. The white edges are in top-down order and the red and green edges are in bottom-up order. Again, we know that all edges incident on nodes below $u_i$ in $L(x)$ have already been processed.

\subsubsection{Processing white non-singleton edges}
For processing white non-singleton edges, we will introduce some extra information that should be stored in $\mathcal{T}$, but only while processing the white edges incident on $u_i$.

The following information is stored at each node $z \in {anc(l_1) \cup anc(l_2) \cup ... }$ as they are being visited the first time.

\begin{itemize}
	\item $rm(z)$: This is the heaviest of the agreement matching of $m(z'), z' \in rfringe(z)$.
		\subitem $rm(z) = max_{z'\in rfringe(z)} m(z')$
	\item $rx(z)$: This is the heaviest of the agreement matching of $x(z'), z' \in rfringe(z)$.
		\subitem $rx(z) = max_{z'\in rfringe(z)} x(z')$
	\item $ry(z)$: This is the heaviest of the agreement matching of $y(z'), z' \in rfringe(z)$.
		\subitem $ry(z) = max_{z'\in rfringe(z)} y(z')$
	\item $ag(z)$: This is the heaviest of the edges $g(z'), z' \in anc(z)$.
		\subitem $ag(z) = max_{z'\in anc(z)} g(z')$
	\item $gr(z)$: This is the heaviest single red edge or green-red proper crossing where the red edge is in the subtree $\mathcal{T}(z')$ for some $z' \in rfringe(z)$. The green edge forms a proper crossing with all red edges in $\mathcal{T}(z')$.
		\subitem $gr(z) = max_{z'\in rfringe(z)} (max_{z''\in anc(z')} g(z'')) + r(z')$, if $r(z')$ exists.
\end{itemize}

Searching for leaves in $\mathcal{T}$ is done by starting from the root of $\mathcal{T}$. Thus finding the above values can be done in constant time for each node $z$ in the following way, where $p(z)$ is the parent node of $z$:
\begin{itemize}
	\item $ag(z) = max\{ag(p(z)), g(z)\}$
	\item If $z$ is the left child of its parent $p(z)$ and $z'$ is the right child, then
		\subitem $rm(z) = max\{rm(p(z)), m(z')\}$
		\subitem $rx(z) = max\{rx(p(z)), x(z')\}$
		\subitem $ry(z) = max\{ry(p(z)), y(z')\}$
		\begin{equation*}
		\begin{aligned} gr(z)=max
		\begin{cases}
		gr(p(z))
		\\
		ag(z') + r(z') & \text{if $r(z')$ exists}             
		\end{cases}
		\end{aligned}
		\phantom{\hspace{3.5cm}}
		\end{equation*}
	\item Otherwise if $z$ is the right child of its parent $p(z)$, then
	\subitem $rm(z) = rm(p(z))$
	\subitem $rx(z) = rx(p(z))$
	\subitem $ry(z) = ry(p(z))$
	\subitem $gr(z) = gr(p(z))$
\end{itemize}

Processing the white edges in top-down order ensures that when processing a white edge $e_{j'}=(u_i,v_j)$, no other edge incident on $u_i$ and on a node in $R(x)$ below $v_j$ has been processed. This means that the agreement matchings $rm(l_{j'}), rx(l_{j'}), ry(l_{j'})$ and $gr(l_{j'})$ only contains edges dominated by $e_{j'}$.

When processing the topmost edge $e_1$, we will search for leaf $l_1$ in $\mathcal{T}$ with a standard search while storing the above values at the nodes of $anc(l_1)$. Now as when processing singleton white edges, we need to find the LWAMs containing only edges dominated by $e_1$ and then adding $e_1$ as the topmost white edge. These LWAMs are exactly the ones we have now stored at $l_1$: $rm(l_1), rx(l_1), ry(l_1)$ and $gr(l_1)$. The largest weight agreement matching with $e_1$ as topmost white edge is therefore the heaviest of these agreement matchings with $e_1$ added as topmost white edge. Now the values $m(z), z \in anc(l_1)$ needs to be updated with this LWAM, but since the remaining edges incident on $u_i$ might have some of the same ancestors, we will instead update the values while searching for these edges.

When searching for a leaf $l_{j'}, j' > 1$, starting at $l_{j'-1}$, we first go up through the tree until reaching the first node which is an ancestor of both $l_{j'}$ and $l_{j'-1}$ and afterwards search for $l_{j'}$ in the standard way while updating values at the nodes visited as before. While going up through the tree, we can update the nodes visited with the LWAM found when processing the previous edge $e_{j'-1}$, since these nodes are all ancestors of $l_{j'-1}$. Reaching a node $z$ where $m(z)$ is heavier than the agreement matching we are currently updating with, means that it is an agreement matching found when processing one of the previous edges, so the following nodes to be visited should instead be updated with that matching.

Having processed the last edge, we will continue to the root of $\mathcal{T}$ in order to update the remaining ancestors.

\subsubsection{Processing red and green non-singleton edges} 
The only extra information that should be stored in $\mathcal{T}$ while processing red and green edges incident on $u_i$ is $ag(z)$ for each node $z \in {anc(l_1) \cup anc(l_2) \cup ... }$, where $ag(z)$ is defined as when processing white non-singleton edges and found the same way while searching for leaves in $\mathcal{T}$.

The procedure is similar to processing white non-singleton edges. The leaves of $\mathcal{T}$ are searched for using sequential search, but starts from the bottommost edge. In the end, we return to the root of $\mathcal{T}$.
\\
\\
As when processing singleton edges, adding a red edge $e_{j'}$ to $\mathcal{T}$ might changes the values $y(z), g(z)$ and $r(z), z \in anc(l_{j'})$ and $g(z'), z' \in lfringe(l_{j'}) \cup rfringe(l_{j'})$. For $z \in anc(l_{j'})$, $y(z)$ should be updated:
\begin{equation*}
\begin{aligned}
y(z)=max
\begin{cases}
y(z)
\\
(max_{z' \in anc(z)} g(z')) + r(z) & \text{if both edges exist}             
\end{cases}
\end{aligned}
\phantom{\hspace{6cm}}
\end{equation*}
\begin{equation*}
\begin{aligned}
\hspace{0.8cm}=max
\begin{cases}
y(z)
\\
ag(z) + r(z) & \text{if both edges exist}             
\end{cases}
\end{aligned}
\phantom{\hspace{6cm}}
\end{equation*}

\noindent For each $z \in lfringe(l_{j'}) \cup rfringe(l_{j'})$, $g(z)$ should be updated to $max_{z' \in anc(z)} g(z') = ag(z)$ and finally for $z \in anc(l_{j'})$, $g(z)$ should be reset and $r(z)$ updated to $max\{r(z), e_{j'}\}$.

Having stored $ag(z)$ at each node $z$ visited while searching, all of the above values can be updated in constant time for each node while going up through the tree.

Notice that when resetting $g(z)$ for some node $z$, $ag(z)$ also needs to be reset since $z$ might be visited again and the value of $ag(z)$ is no longer valid.
\\
\\
Adding a green edge $e_{j'}$ to $\mathcal{T}$ might changes the values $g(z), z \in lfringe(l_{j'})$ and $x(z), z \in anc(l_{j'})$. For $z \in lfringe(l_{j'})$, $g(z)$ is updated to $max\{e_{j'}, max_{z' \in anc(z)} g(z')\} = max\{e_{j'}, ag(z)\}$. Since $ag(z)$ is stored at each node $z$ visited while searching, $g(z), z \in lfringe(l_{j'})$ is updated in constant time for each node while going up through the tree. For $z \in anc(l_{j'})$, $x(z)$ should be updated to $max\{x(z), e_{j'} + max(r(z'))\}$, where $max(r(z'))$ maximizes over the vertices $z' \in lfringe(l_{j'})$ which are descendants of $z$. This is also done in constant time for node $z \in anc(l_{j'})$ while going up through the tree by keeping track of the heaviest red edge $hre$ of $r(z'), z' \in lfringe(l_{j'})$, where $z'$ is a descendant of $z$.

If $z$ is the first node which is an ancestor of both $l_{j'}$ and $l_{j'+1}$ (assuming that $l_{j'+1}$ exists), we will store $e_{j'}$ at $z$ as $ge(z)$. Now when going up through the tree after having reached leaf $l_{j'}$, we will also keep track of which green edge $hge$, incident on $u_i$ is the heaviest that has been added to the subtree rooted at the current node $z$. When at $l_{j'}$, $hge$ is obviously $e_{j'}$, and when at a node $z \in anc(l_{j'})$ it is simply the heaviest edge of $hge$ and $ge(z)$. $x(z)$ can now be updated to $max\{x(z), hge + hre\}$ in constant time.

\subsubsection{Extracting the Largest Weight Agreement Matchings}
First, we need to compute the LWAM containing only edges from $u_i$ and nodes below $u_i$ in $L(x)$, for each $u_i \in L(x)$. For any $u_i \in L(x)$, all the information needed to compute such a matching is stored at the root $r$ of $\mathcal{T}$ after having processed the edges incident on $u_i$ and before processing edges incident on the above nodes in $L(x)$. The LWAM is extracted in the following way:

If the LWAM contains a white edge it is simply given by $m(r)$. If the LWAM is a proper crossing, but not a single red edge, it is given by $x(r)$. And finally, if the LWAM is a single red edge, it is given by $r(r)$. Therefore the LWAM is the heaviest of these three matchings. This LWAM is saved, such that it can be retrieved in constant time given $u_i$ and $x$. 

For each $v_j \in R(x)$, we also need to compute the LWAM containing only edges incident on $v_j$ and nodes below $v_j$ in $R(x)$. For leaf $v_j \in R(x)$, such a matching will correspond to the MAST between $T_1$ and $T_2(v_j)$, so this is the LWAM that needs to be stored at $v_j \in T_2$. The LWAMs are extracted from $\mathcal{T}$, after all edges in $G(x)$ have been processed. Each node of $R(x)$ is processed in order, starting from the bottom:

For node $v_j \in R(x)$, let $l$ be the leaf in $\mathcal{T}$ corresponding to $v_j$ and let $M(v_j)$ be the LWAM containing only edges incident on $v_j$ and nodes below $v_j$ in $R(x)$. $M(v_j)$ is either equal to $M(v_{j+1}), j < |R(x)|$, or it is the LWAM having a white edge incident on $v_j$ as the topmost edge, or it is the largest proper crossing having a red edge incident on $v_j$, or it is a single green edge incident on $v_j$. Thus, $M(v_j)$ is given by:
\begin{equation*}
\begin{aligned}
M(v_j)=max
\begin{cases}
M(v_{j+1})						& \text{if $j < |R(x)|$}
\\
m(l)
\\
y(l)
\\
max_{z \in anc(l)} g(z) + r(l)	& \text{if r(l) exist}    
\\
x(l)         
\end{cases}
\end{aligned}
\phantom{\hspace{6cm}}
\end{equation*}
After having computed the LWAMs for all matching graphs, the LWAM corresponding to the MAST between $T_1$ and $T_2$ can be retrieved in constant time given the root $u_1$ of $T_1$ and the root $v_1$ of $T_2$. This LWAM was extracted from the search tree for the matching graph $G(v_1)$ and is the LWAM containing edges from $u_1$ and nodes below $u_1$ in $L(v_1)$ which is all possible edges and therefore corresponds to the MAST between $T_1$ and $T_2$.

\subsubsection{Analysis}
\label{lwam_analysis}
In \cite{nlogn}, Cole et. al. proves two important lemmas:

\begin{Lemma}
	Consider matching graph $G(x)$. Then
	
	$\sum_{i|d_x(u_i)>1} d_x(u_i)log\dfrac{nsav(x)}{d_x(u_i)} \le \sum_{i|d_x(u_i)>1} d_x(u_i)log\dfrac{n}{m_i}$
	\label{analysisLemma1}
\end{Lemma}
\begin{Lemma}
	Consider node $u_i \in \pi$. Then
	
	$\sum_{x \in X | d_x(u_i)>1} d_x(u_i) = O(m_i)$
	\label{analysisLemma2}
\end{Lemma}

\noindent Consider the matching graph $G(x)$ with search tree $\mathcal{T}$ and a node $u_i \in L(x)$. If $d_x(u_i) > 1$ and $L$ is the ordered set of leaves in $\mathcal{T}$ that corresponds to the nodes in $R(x)$ incident on the edges from $u_i$, then it follows from the properties of the search tree and Lemma \ref{analysisLemma1} that the amount of nodes in the search tree that is visited when searching for the leaves of $L$ is $O(d_x(u_i)log(\dfrac{n}{m_i}))$.

Considering node $u_i \in \pi$, it follows from Lemma \ref{analysisLemma2} that the amount of nodes visited in all search trees over all matching graphs $G(x)$ with $d_x(u_i) > 1$ is $O(m_ilog\dfrac{n}{m_i})$.

For each node $u_i \in \pi$ over all matching graphs $G(x)$ with $d_x(u_i) = 1$, it turns out that the amount of nodes visited in all search trees is also $O(m_ilog\dfrac{n}{m_i})$. This can be seen from the analysis in \cite{nlogn}.

Thus the total amount of nodes visited in all search trees, for any node $u_i \in \pi$, is $O(m_ilog\dfrac{n}{m_i})$.

\subsubsection{Time Complexity}
When processing an edge, whether it is a singleton or a non-singleton edge, each node of the search tree that is visited, is visited no more than a constant number of times. When visiting a node, only a constant amount of work is performed, so for any node $u_i \in \pi$, the time taken to process $u_i$ over all matching graphs must be $O(m_ilog\dfrac{n}{m_i})$.

Extracting the LWAM for edges from $u_i$ and nodes below $u_i$ in $L(x)$, for each $u_i \in L(x)$ takes constant time after having processed $u_i$. Extracting the LWAM for edges from $v_j$ and nodes below $v_j$ in $R(x)$, for each $v_j \in R(x)$ is done through a single traversal of $\mathcal{T}$ in $O(|R(x)|)$ time. This means that the total time for computing the largest weight agreement matchings is $O(\sum_{i=1}^p m_ilog\dfrac{n}{m_i}) \le O(nlogn)$.
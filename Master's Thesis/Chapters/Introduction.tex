\chapter{Introduction}
\label{ch:intro}
The Maximum Agreement Subtree problem (MAST) is concerned with mutual information between rooted trees, and is defined as such: Given two rooted trees, $T_1$ and $T_2$, created over the same leaf-set $\{1,2,3...,n\}$, determine the largest possible subset of leaves inducing an agreeing subtree of $T_1$ and $T_2$. For a set of leaves to induce an agreeing subtree for $T_1$ and $T_2$, the subtrees restricted to the set of leaves must be isomorphic, implying structural equivalence.
\\

Let us start by motivating the interest in MAST by giving an example of its application. Suppose that we are interested in inspecting the relationship between DNA obtained from different animal species. This is typically done by the use of  Hierarchical Clustering or Neighbour Joining to construct evolutionary trees. However, finding the true evolutionary tree is often an elusive task, and evidence is required to support any suggested tree topology. Finding the Maximum Agreement Subtree will present the information that both trees agree on, which makes the information more reliable, given that multiple sources support it. 

The MAST problem applies to all trees, but we will focus on the rooted, binary trees given that the motivation for the problem is primarily rooted in biology and linguistics, where these trees are most common.

Several different algorithms have been developed for solving the MAST problem with different time complexities. One of these is the algorithm described by Cole et. al.\ \cite{nlogn} which theoretically has a time complexity of $O(nlogn)$, where $n$ is the number of leaves in each of the two input trees.

In the paper, we will specifically focus on this algorithm. We will give a detailed description of how the algorithm works and how it can be implemented. We will also walk through the algorithm described by Goddard et. al.\ \cite{nsquared} with time complexity $O(n^2)$ and compare the two algorithms in order to clarify strengths and weaknesses of each in theory and in practice.

\section{Thesis Structure}
The thesis is structured as follows. Chapter 2 gives some practical information about the programs we have implemented. In chapter 3 we explain the most basic solution to the MAST problem, which is used as a verification tool for the other algorithms in the paper. In chapter 4 we walk through the $O(n^2)$ algorithm for solving the MAST problem described by Goddard et. al. \cite{nsquared}. In chapter 5 we focus on the $O(nlogn)$ algorithm described by Cole et. al. \cite{nlogn}. We specifically go though each step of the algorithm and describe both the time and space complexity by the end of the chapter. Chapter 6 describes how we used the naive algorithm to verify the correctness of our implementations of the $O(n^2)$ and $O(nlogn)$ algorithms. Chapter 7 covers the experiments conducted in order to verify the theoretical claims regarding time and space complexity. Finally, Chapter 8 briefly sums up the results of our work, and what future work might include.

\section{Division of Labour}
Throughout the development of the master thesis, Nikolaj has had a great deal of illness, which changed our initial plans of how the work should be shared between us. Initially, we worked together in understanding and implementing the algorithms and writing the thesis. However due to Nikolajs illness, the last part of the thesis was primarily conducted by Thomas.

Working on the master thesis comprised the following:

\subsubsection{Reading and Understanding the Articles}
Early on, we worked together in reading and understanding the articles to a point that made it possible for us to start implementing the algorithms.

\subsubsection{Implementing the Algorithms}
The first algorithms to implement was the $O(n^2)$ algorithm and the naive algorithm. This was done together. We worked together in planning and designing the implementation of the $O(nlogn)$ algorithm and did the first parts of the implementation together. Completing the implementation was done by Thomas.

\subsubsection{Testing Algorithms and doing Experiments}
Testing the correctness of the algorithms was done together. We designed the first experiments together, but the final experiments were designed and carried out by Thomas. We discussed the results of all experiments together.

\subsubsection{Structuring and Writing the Thesis}
Structuring the thesis was done together. The chapters describing the naive and the $O(n^2)$ algorithm was primarily written by Nikolaj, while the chapters describing the $O(nlogn)$ algorithm and the experiments was primarily written by Thomas. The rest of the thesis was done by both of us.





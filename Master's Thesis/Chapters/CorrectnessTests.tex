\chapter{Testing Correctness}
Having implemented the algorithms for the Maximum Agreement Subtree Problem, we wanted to test the correctness of each of them. Since the naive algorithm simply tries all possible combinations of leaves to prune from the two input trees and picks the one giving the largest agreement subtree, we believe it to be correct though very inefficient. By manually verifying results of the algorithm for small input trees, we convinced ourselves that the implementation did not contain any errors.

The $O(n^2)$ algorithm could now be tested against the naive algorithm by running the two algorithms on the same input trees and verifying that the result trees were of the same size. Because of the time complexity of the naive algorithm, we could only to this for small input trees. This was done for random input trees of sizes between 1 and 50. The $O(nlogn)$ algorithm was tested against the $O(n^2)$ algorithm for the same type of input trees, but of sizes between 1 and 10000. The fact that the $O(n^2)$ algorithm had already been tested and that it is very unlikely that the two algorithms would fail on the same input giving the same false result, indicates that both algorithms are correct when passing the test.

Also the algorithm using the MLIS solution has been verified. This algorithm was tested against the $O(nlogn)$ algorithm.
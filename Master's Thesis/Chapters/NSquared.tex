\chapter{\todo{The NSquared algorithm}}
Goddard et. al.\cite{nsquared} describes a $O(n^2)$ algorithm for finding the largest agreement subtree for two rooted binary trees. Given two trees T and U of size m and n, the idea is to iteratively find the largest agreement subtree and its size for every pair of subtrees from T and U. This can be done in quadratic time by using Lemma 1. 

\begin{Lemma}
	Let $T_a$ be a tree rooted at vertex a with the children b and c; And let $T_w$ be rooted at w with children x and y. We now define $\#(T_a,T_x)$ as the size of the maximum agreement subtree of $T_a$ and $T_w$.  $\#(T_a,T_x)$ is given by the maximum of the following six numbers: \#$(T_b,T_x)+(T_c,T_y)$,
	\#$(T_b,T_y)+(T_c,T_x)$,
	\#$(T_a,T_x)$,
	\#$(T_a,T_y)$,
	\#$(T_b,T_w)$,
	\#$(T_c,T_w)$
\end{Lemma}

What Lemma 1 essentially states is twofold. First, if the MAST contains children from all subtrees (a,b,x,y), then its size is given by the largest combination of these. Secondly, if the MAST does not include all subtrees, then its size is given by the largest combination of subtrees excluding at least one of them.    
This gives rise to a recursive solution to finding the size of the maximum agreement subtree. The recursive function is defined as follows: 

\begin{figure}
	\begin{equation*}
	\begin{aligned}
	f(T_a,T_w)=Max
	\begin{cases}
	f(T_b,T_x)+f(T_c,T_y) & \text{if Type($T_a$)=Type($T_w$)=Internal Node}
	\\
	f(T_b,T_y)+f(T_c,T_x) & \text{if Type($T_a$)=Type($T_w$)=Internal Node}
	\\
	f(T_a, T_x)           & \text{if Type($T_w$)=Internal Node}
	\\
	f(T_a, T_y)           & \text{if Type($T_w$)=Internal Node}
	\\
	f(T_b, T_w)           & \text{if Type($T_a$)=Internal Node}
	\\
	f(T_c, T_w)           & \text{if Type($T_a$)=Internal Node}
	\\
	1 	                  & \text{if Type($T_a$)=Type($T_w$)=Leaf  $\land$  $T_a$=$T_w$}
	\\
	0                     
	\end{cases}
	\end{aligned}
	\phantom{\hspace{6cm}}
	\end{equation*}
	\caption{Recursive MAST size function}
	\label{Fig:Function1}
\end{figure}


Like many similar recursive definitions in bioinformatics, we run into the problem that the recursive function computes the same partial results multiple times. This problem is rectified by the method of dynamic programming. Specifically, we wish to store the partial results in a $|T_a| \times |T_w|$ table. We seek to fill out the table with our partial results, and end up with the maximum size in the bottom right corner.
Looking at the function in Figure \ref{Fig:Function1} we see that the partial result for each tree node is dependent on the results of its children. This implies that we must list the table nodes in postorder, thereby ensuring that the partial results, on which a given node is dependent, has already been calculated when we reach it. 
By introducing such a table we also introduce a requirement of $O(n^2)$ space and time, making it a quadratic algorithm. Simple techniques for reducing the space requirement, like only storing a few number of rows in the table at a time, cannot be applied in this case, since the computation of a new row may depend on all the previous rows in the worst case.  
\\
We will in the following sections show how we extend this method for calculation the size of the MAST to computing the actual tree.

\subsection{Algorithm example}
Two example trees have been provided in Figure \ref{Fig:Binary1}. We now wish to compute the size of their maximum agreement subtree using the agorithm desribed above. For this reason, we create a $15 \times 15$ table containing the nodes of each tree in postorder. Using the function desribed in (3.1), we can now fill out the table by starting from the topmost row and going towards the bottom. The resulting table can be seen in \ref{Table:Table1}, where we from the bottom right corner than see that the maximum size of the MAST is 6. By looking at the trees for a little while, one can come to the conclusion that 6 is indeed correct, since removing only one leaf cannot produce an agreement tree in this case, while removing $leaf_1 and leaf_7$ will produce an agreement tree of size 6. 

\begin{figure}
	
	\begin{itemize}
		\item[] a. \Tree [.A [.B [.C leaf_1 leaf_2 ] [.D leaf_3 leaf_4 ] ].B [.E [.F leaf_5 leaf_6 ] [.G leaf_7 leaf_8 ] ].E ].A
		
		b. \Tree [.H [.I [.J leaf_1  leaf_8 ] [.L leaf_5 leaf_6 ] ].I [.M [.N leaf_2 leaf_7 ] [.F leaf_3 leaf_4 ] ].M ].H
	\end{itemize}	
	
	\caption{Two example binary trees for MAST comparison}
	\label{Fig:Binary1}	
\end{figure}


\begin{table}[]
	\centering
	\begin{tabular}{|c|c|c|c|c|c|c|c|}
		\hline
		\textbf{}        & \textbf{leaf\_1} & \textbf{leaf\_2} & \textbf{C} & \textbf{.....} & \textbf{G} & \textbf{E} & \textbf{A} \\ \hline
		\textbf{leaf\_1} & 1                & 0                & 1          & ...            & 0          & 0          & 1          \\ \hline
		\textbf{leaf\_8} & 0                & 0                & 0          & ...            & 1          & 1          & 1          \\ \hline
		\textbf{J}       & 1                & 0                & 1          & ...            & 1          & 1          & 2          \\ \hline
		\textbf{...}     & ...              & ...              & ...        & ...            & ...        & ...        & ...        \\ \hline
		\textbf{F}       & 0                & 0                & 0          & ...            & 0          & 0          & 2          \\ \hline
		\textbf{M}       & 0                & 1                & 1          & ...            & 1          & 1          & 3          \\ \hline
		\textbf{H}       & 1                & 1                & 2          & ...            & 2          & 3          & 6          \\ \hline
	\end{tabular}
	
	\label{Table:Table1}
	\caption{Score Matrix for the Trees in Figure 3.1}
\end{table}




\subsection{Computing the MAST}
Extending the solution for computing the size of the MAST to include the computation of the actual MAST is fairly simple in this case. We found two different ways of going about it. 
\\
\textbf{Option 1:}



\begin{lstlisting}[language=Java, caption=Java pseudo code for BackTracking]
Function RecBackTrack(Tree Tree1, Tree Tree2)
{
	int Score = ScoreMatrix[Tree1][Tree2];

	if(Type(Tree1) = Type(Tree2) = InternalNode)
	{
		int Score1 = ScoreMatrix[Tree1.Child1][Tree2.Child1];
		int Score2 = ScoreMatrix[Tree1.Child2][Tree2.Child2];
		int Score3 = ScoreMatrix[Tree1.Child1][Tree2.Child2];
		int Score4 = ScoreMatrix[Tree1.Child2][Tree2.Child1];

		if(Score = Score1 + Score2) {
			Tree subTree1 = RecBackTrack(Tree1.Child1, Tree2.Child1);
			Tree subTree2 = RecBackTrack(Tree1.Child2, Tree2.Child2);
			//Return internal node with subTree1 and subTree2 as children
		}
		else if (Score1 = Score3 + Score4) {
			Tree subTree1 = RecBackTrack(Tree1.Child1, Tree2.Child2);
			Tree subTree2 = RecBackTrack(Tree1.Child2, Tree2.Child1);
			//Return internal node with subTree1 and subTree2 as children
		}
	}

	else if (Type(Tree1) = InternalNode)
	{
		int Score1 = ScoreMatrix[Tree1.Child1][Tree2]; 
		int Score2 = ScoreMatrix[Tree1.Child2][Tree2];

		if (Score = Score1) {
			return RecBackTrack(Tree1.Child1, Tree2);
		}
		else if (Score = Score2) {
			return RecBackTrack(Tree1.Child2, Tree2);
		}
	}

	else if (Type(Tree2) = InternalNode)
	{
		int Score1 = ScoreMatrix[Tree1][Tree2.Child1]; 
		int Score2 = ScoreMatrix[Tree1][Tree2.Child2];

		else if (Score = Score1) {
			return RecBackTrack(Tree1, Tree2.Child1);
		}
		else if (Score = Score2) {
			return RecBackTrack(Tree1, Tree2.Child2);
		}
	}

	else if (Type(Tree1)= Type(Tree2) = Leaf) {
		return Tree1 = Tree2 ? Tree1 : null; 
	}	
}
\end{lstlisting}



\todo{\dots}

\section{Implementation}
We implemented the algorithm in java (using the forrester \cite{?} library to represent the trees?). We computed the agreement subtrees for each pair of subtrees in the two trees by doing a postorder traversal of the first tree and for each node did a postorder traversal of the second tree. For each pair of nodes $a$ and $w$ we computed the agreement subtree $A_{a,w}$ for the two subtrees $T_a$ and $T_w$ having respectively $a$ and $w$ as roots. For such a pair of nodes there are three cases.

The first case is that $a$ and $w$ are both leaves. If the leaves are equal, i.e. they have the same name, then $A_{a,w}$ will be the tree consisting of exactly one leaf with that name. Otherwise $A_{a,w}$ is empty.

The second case is that we have a leaf and an internal node. Let's say $w$ is the internal node having $x$ and $y$ as children. If the leaf corresponding to $a$ is contained in $T_w$, it will also be contained in either $T_x$ or $T_y$ and $A_{a,w}$ will be the same as either $A_{a,x}$ or $A_{a,y}$. Since we do a postorder traversal of the trees, $A_{a,x}$ and $A_{a,y}$ have already been computed and $A_{a,w}$ can just be set to the largest of the two.

The third case is that both $a$ and $w$ are internal nodes, where $a$ have children $b$ and $c$ and $w$ have children $x$ and $y$. In this case we can use Lemma 1. Again, all the agreement subtrees to consider have already been computed. If the largest subtree is either $A_{b,x} + A_{c,y}$ or $A_{b,y} + A_{c,x}$, $A_{a,w}$ will be the tree having the two subtrees as children.

All these agreement subtrees were stored in a matrix together with their sizes. The last cell in the matrix would then at the end contain the agreement subtree for the two input trees.
\\
\\
How did we verify the algorithm? E.g. bruteforce MAST and compare results.

Show trees and MAST outputted by the program.

\todo{\dots}
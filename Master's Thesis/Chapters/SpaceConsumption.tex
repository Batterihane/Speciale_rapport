As we will later explain from our experiments, it turns out that the amount of space used by the algorithm is an important factor for the runtime when running the algorithm in practice. We will therefore in this section try to get an overview of how much space is needed by the algorithm.

We will walk through each step of the algorithm and give an analysis of the amount of space that is needed.

\subsection{Centroid Decompositions}
When creating centroid decompositions, each node of $T_1$ and $T_2$ will be part of at most one centroid path which is all that needs to be stored. The amount of space needed is therefore $O(n)$.

\subsection{Induced Subtrees}
\subsubsection{Preprocessing}
Before inducing subtrees, $T_2$ needs to be preprocessed for finding LCAs in $O(1)$ time. This requires storing a constant amount of data at each node of $T_2$ and creating a new tree of size $O(n)$ where a constant amount of data is also stored at each node.

Preprocessing $T_2$ for inducing subtrees in linear time is also done by storing a constant amount of data at each node of $T_2$, giving a space consumption of $O(n)$.

\subsubsection{Inducing Subtrees}
Inducing a subtree of $T_2$ from a set of leaves $L$ only requires storing data for each node of the subtree which will have size $O(|L|)$.

For each side tree $M_i, 1 \le i \le p-1$ of $\pi$, the subtree $S_i$ is induced from the leaves of $T_2$ that are twins of the leaves of $M_i$. The total amount of leaves used to induce subtrees is therefore $\sum_{i=1}^{p-1} m_i < n$ and the space needed is $O(n)$.
\\
\\
In order to create the sets of leaves used to induce subtrees, the leaves of the side trees of $\pi$ should be sorted, which can be done using bucket sort, giving a space consumption of $O(n)$.

\subsection{Matching Graphs}
Recall that a matching graph $G(x)$ consists of two sets of nodes $L(x)$ and $R(x)$ with edges between them. Only a constant amount of data needs to be stored at each node and each edge. As mentioned earlier, the total number of edges in all matching graphs has an upper bound of $O(nlogn)$ and since $|L(x)| \le n$ and $|R(x)| \le n$, the total amount of space needed for the matching graphs must be $O(nlogn)$.

\subsection{Agreement Matchings}
\subsubsection{The Weighted Search Tree}
For each matching graph $G(x)$ a binary search tree is created. The search tree gets a leaf for each node in $R(x)$, so the amount of space needed for a single search tree is $O(|R(x)|)$. Each node of $T_2$ can only be contained in one centroid path and the sets $R(x), x \in X$ contain nodes from distinct centroid paths so the amount of space needed for all search trees is $O(\sum_{x \in X} |R(x)|) = O(n)$.

\subsubsection{Largest Weight Agreement Matchings}
Recall that for each matching graph $G(x)$, we compute and store the LWAM containing only edges from $u_i$ and nodes below $u_i$ in $L(x)$, for each $u_i \in L(x)$, and we compute and store the LWAM containing only edges from $v_j$ and nodes below $v_j$ in $R(x)$, for each $v_j \in R(x)$. The LWAMs are constructed while processing the edges of $G(x)$ and saved in the search tree. When updating a node in the search tree, only a constant amount of information is added. Either an edge or a proper crossing is updated, or an agreement matching is updated which is either a proper crossing or a white edge and a pointer to an agreement matching. So the amount of space needed to store the LWAMs is proportional to the number of times a node is updated.

In section \ref{lwam_analysis} we concluded that the total amount of nodes visited when searching for leaves in the search trees, for any node $u_i \in \pi$, is $O(m_ilog\dfrac{n}{m_i})$ and these nodes are visited only a constant number of times. So the amount of space needed for LWAMs must be  $O(\sum_{i=1}^p m_ilog\dfrac{n}{m_i}) \le O(nlogn)$.

\subsection{Creating the MAST}
Creating the final MAST requires no more space than the size of the tree which can't be larger than $n$, so the amount of space used is $O(n)$.

\subsection{\todo{Total}}
We will show that the total amount of space needed for storing the LWAMs that are recursively constructed is $O(nlogn)$. We can inductively assume that each recursive call $computeLWAM(M_i, S_i)$ uses $O(m_ilogm_i)$ space, $1 \le i \le p-1$. Creating centroid decompositions, inducing subtrees and creating the final MAST uses $O(n)$ space and creating matching graphs and LWAMs uses $O(nlogn)$ space, so the total amount of space used by the algorithm is $O(n) + O(nlogn) + O(\sum_{i=1}^{p-1} m_ilogm_i) = O(nlogn)$.

Table \ref{spaceTable} gives an overview.
\begin{table}[]
	\centering
	\begin{tabular}{l|l}
		Creating Centroid Decompositions & O(n)     \\
		Inducing Subtrees                & O(n)     \\
		Construct LWAMs recursively		 & O(nlogn) \\
		Creating Matching Graphs         & O(nlogn) \\
		Computing LWAMs                  & O(nlogn) \\
		Creating the MAST                & O(n)     \\ \hline
		Total                            & O(nlogn)
	\end{tabular}
	\caption{Runtime}
	\label{spaceTable}
\end{table}





